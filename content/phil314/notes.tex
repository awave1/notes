\documentclass{article}
\usepackage[utf8]{inputenc}
\usepackage{hyperref}
\usepackage[parfill]{parskip}
\usepackage{graphicx}
\graphicspath{ {./} }

\hypersetup{
    colorlinks,
    citecolor=black,
    filecolor=black,
    linkcolor=black,
    urlcolor=black
}

\title{PHIL 314 - Information Technology Ethics}
\author{Artem Golovin}

\begin{document}

\maketitle
\tableofcontents
\newpage

% Week 1: Privacy
% Week 2: Intellectual Property

\section{Tuesday, May 7th 2019}

% Quiz 1
% T
% T
% F (privacy as secrecy is Posner's view)
% F (?)
% T
% F

\subsection{Tony Doyle, "Posner on Privacy" (2013)}

Richard Posner is a leading contemporary critic of privacy. Posner
is highly skeptical of most appeals to privacy, characterizing them as selfserving
attempts to keep discrediting, embarrassing, or inconvenient facts
from others.

\subsubsection{What is Posner's economic analysis of privacy?}

It is economic because Posner sees people as potential consumers. As a consumer, you would want to find possible cons of the product you want to buy, so you will do anything to find those shortcomings. Same idea applies to privacy. Social relations create tension/reputation as an asset. Posner wants to value efficiency the most, he is \textbf{consequentialist}. Since social sphere is a product market, people would want to hide their flaws. TODO.

Privacy as Manipulation - appeals to a right to privacy are motivated by the desire to mislead or manipulate others. Privacy as Secrecy - the history of drunk driving, marital discord, evil temper, criminal records, etc. Therefore, when you're appealing to privacy, that most likely mean you have something to hide.

\subsubsection{In what context is privacy is allowed?}

Posner argues that you do have a right to privacy when a person is at home or privacy of private conversation. Domestic business at home should be protected, because otherwise blackmail is possible. Ordinary conversation: social value in frank communications; fostering free exchange of ideas.

\subsection{Criticisms on Posner's View}

\subsubsection*{Posner's transparency views}

For example, in the job market, if potential employer wants to look up information about you online, it will result in chilling effect and will make you as a potential employee to get away from that employer. Employers could also discriminate in hiring if the potential employer is biased.

\subsubsection*{Justified cases of concealment}

Not all privacy appeals are dishonest. Concealing is not the same as manipulation. Calculation of profit and loss is empirical question, we can't just presume that Posner's idea is correct, we have to look at the individual cases and look at the loss. Also Doyle mentions that no intention to harm can actualy cause the harm. For example, selling products online and targeting specific user group might cause harm to some individuals that did not want to reveal such information. Most importantly, Posner fails to acknowledge the downside of digitized information.

\subsection{Doyle's privacy advocacy}

\subsubsection{Classic panopticon vs. Modern panopticon}

“The principle is central inspection. You can do central inspection by CCTV. You don’t need a round building to do it. Monitoring electronic communications from a central location, that is panoptic. The real heart of Bentham’s panoptic idea is that there are certain activities which are better conducted when they are supervised.” (\url{https://www.theguardian.com/technology/2015/jul/23/panopticon-digital-surveillance-jeremy-bentham})

\subsubsection{New threat posed by the digital age}

Information about us is digital, therefore it is easily stored on servers/computers, it's almost impossible to get rid of all information that is already stored. Digital information can be traced easily and it can also be manipulated in various ways. Our mundane actions leave a lot of digital records. Unintruisive gathering of information becomes intruisive. Companies like Google can know tremendous information about you, but you might know little about Google. The asymmetry of information implies a "loss of consumer autonomy" and "loss of control of the information".

\subsection{Surveillance. Post 9/11 Privacy views}

// TODO

Since 9/11, Posner shifted his position.

\subsection{Personally Identifiable Information (PII)}

The concept of Personally Identifiable Information (PII) is central in privacy regulation. No PII in the collected data, then where is the harm in diclosing personal information. The scope of information requires privacy protection. Due to technological advances, it is somewhat easy to get some PII based on "public" information.

\section{Thursday, May 9th 2019}

\subsection{Background: Post 9/11}

NSA starts mass surveillance programs to collect personal information so that they have an ability to prevent and detect possible threats of terrorism.


\subsection*{Variants of the `Nothing To Hide'}

\begin{enumerate}
    \item I don't have anything to hide
    \item All law-abiding citizens should have nothing to hide
    \item Posner's version \#1: watching without watchers (// TODO)
    \item Posner's version \#2: security outweighs privacy (// TODO)
\end{enumerate}

\textbf{Snowden effect}: increase in public knowledge and more reporting about the government surveillance from the public. Since his discovery of NSA surveillance program there has been debate between privacy vs. security.

What are people really afraid of regarding being watched? It's not only about trusting government, anyone can watch you with bad intentions

\subsection{The Nothing to Hide Argument}

\subsection{Conceptualizing Privacy: one-size-fits-all?}

Monistic (reductionistic) understanding of privacy is the essencee/ core characteristic/ common denominator/ necessary and sufficient conditions. Possible common characteristic is intimacy, but it is \textbf{too nerrow}. Right to be let alone is \textbf{too broad} (// TODO). We have to define pluralistic understanding of privacy. 

\section{May 14th, 16th 2019 - Intellectual Property}

\subsection{Monkey Selfie Dispute}

A photographer was travelling in Indonesia. A monkey, "Naruto", took the camera and started taking `selfies'. There have been debates about who owns the copyright to pictures that were taking by the monkey? Debate over the assignment of copyright protection to non-human beings as well as jurisdictional issues in connection with the online publication of the pictures. Main conclusion of the legal case is that non-human beings are not entitled to copyright protection.

\subsection{The Justification of Intellectual Property}

\subsubsection{Background}

World Intellectual Property Organization was created in 1967 to encourage creative activity, to promote the protection of intellectual property throughout the world. Samir Chopra argues that IP is "a culturally damaging and easily weaponized notion". Alternatives to IP: IMPs, GOLEMs. Richard Stallman says it is the problem of overgeneralization.

\textbf{Copyright} protects original works such as books, music, software, sculture, and aspects of computer programs that are embodies or fixed in a tangible medium. Copiright does not protect ideas themselves, it protects \textit{unique expression of ideas}. \textbf{Patent} cover new and useful invections, manufactures, compositions of matter and processes reduced to practice by inventors with requirements of subject matter. Pattents protect realised // TODO

\subsection{What is Property?}

The common claim that abstract objects (or intellectual content) cannot and should not be characterized as property does not work. Ownership does not require physical posession. Something that is not a property does not necessarily lead to the conclusion that it does not need legal protections. // TODO

\subsection{Arguments for IP protection}

\subsubsection*{Effects-Based Arguments}

Determinant of the moral quality of a law is its effects on the relevant index of human well-being. Need to ensure that authors have a sufficient (material) incentive to spend time, effort and labor to create new content. Whenever we talk about desires and incentives to create, these are imperical questions.

\subsubsection*{Arguments From Investment}

A content creator invests something in which a person has morality // TODO

Lockean Argument: "original acquisition of property". Locke talks how creating property and property right by mixing our labour with the unowned land. This argument has two provisos, which are

\begin{enumerate}
    \item Sufficiency: There must be enough of the material object for everyone else to appropriate.
    \item Abscence of Harm: No one may acquire a material object to spoil or destroy it.
\end{enumerate}

Hima mentions two reasons why Lockean argument fails. Lockean argument fails because it applies to materials. E.g. if I swim across an ocean, fence off part of the ocean, then I will meet both provisors, since the ocean wasn's spoiled, and I put a lot of effort to fence it off.

\subsubsection*{Personality argument}

// TODO

All arguments from investment fail since they do not consider the extent to which the lives of others might be worsened by giving an author explusive control over the content of her or his creation.

\subsection{Arguments against IP protection}

// TODO

\section{Tuesday, May 21st 2019}

\subsection{Online hate speech}

\end{document}