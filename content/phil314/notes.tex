\documentclass{article}
\usepackage[utf8]{inputenc}
\usepackage{hyperref}
\usepackage[parfill]{parskip}

\hypersetup{
    colorlinks,
    citecolor=black,
    filecolor=black,
    linkcolor=black,
    urlcolor=black
}

\title{PHIL 314 - Information Technology Ethics}


\begin{document}

\maketitle
\tableofcontents
\newpage

\section{Tuesday, May 7th 2019}

% Quiz 1
% T
% T
% F (privacy as secrecy is Posner's view)
% F (?)
% T
% F

\subsection{Tony Doyle, "Posner on Privacy" (2013)}

Richard Posner is a leading contemporary critic of privacy. Posner
is highly skeptical of most appeals to privacy, characterizing them as selfserving
attempts to keep discrediting, embarrassing, or inconvenient facts
from others.

\subsubsection{What is Posner's economic analysis of privacy?}

It is economic because Posner sees people as potential consumers. As a consumer, you would want to find possible cons of the product you want to buy, so you will do anything to find those shortcomings. Same idea applies to privacy. Social relations create tension/reputation as an asset. Posner wants to value efficiency the most, he is \textbf{consequentialist}. Since social sphere is a product market, people would want to hide their flaws. TODO.

Privacy as Manipulation - appeals to a right to privacy are motivated by the desire to mislead or manipulate others. Privacy as Secrecy - the history of drunk driving, marital discord, evil temper, criminal records, etc. Therefore, when you're appealing to privacy, that most likely mean you have something to hide.

\subsubsection{In what context is privacy is allowed?}

Posner argues that you do have a right to privacy when a person is at home or privacy of private conversation. Domestic business at home should be protected, because otherwise blackmail is possible. Ordinary conversation: social value in frank communications; fostering free exchange of ideas.

\subsection{Criticisms on Posner's View}

\subsubsection*{Posner's transparency views}

For example, in the job market, if potential employer wants to look up information about you online, it will result in chilling effect and will make you as a potential employee to get away from that employer. Employers could also discriminate in hiring if the potential employer is biased.

\subsubsection*{Justified cases of concealment}

Not all privacy appeals are dishonest. Concealing is not the same as manipulation. Calculation of profit and loss is empirical question, we can't just presume that Posner's idea is correct, we have to look at the individual cases and look at the loss. Also Doyle mentions that no intention to harm can actualy cause the harm. For example, selling products online and targeting specific user group might cause harm to some individuals that did not want to reveal such information. Most importantly, Posner fails to acknowledge the downside of digitized information.

\subsection{Doyle's privacy advocacy}

\subsubsection{Classic panopticon vs. Modern panopticon}

“The principle is central inspection. You can do central inspection by CCTV. You don’t need a round building to do it. Monitoring electronic communications from a central location, that is panoptic. The real heart of Bentham’s panoptic idea is that there are certain activities which are better conducted when they are supervised.” (\url{https://www.theguardian.com/technology/2015/jul/23/panopticon-digital-surveillance-jeremy-bentham})

\subsubsection{New threat posed by the digital age}

Information about us is digital, therefore it is easily stored on servers/computers, it's almost impossible to get rid of all information that is already stored. Digital information can be traced easily and it can also be manipulated in various ways. Our mundane actions leave a lot of digital records. Unintruisive gathering of information becomes intruisive. Companies like Google can know tremendous information about you, but you might know little about Google. The asymmetry of information implies a "loss of consumer autonomy" and "loss of control of the information".

\subsection{Surveillance. Post 9/11 Privacy views}

// TODO

Since 9/11, Posner shifted his position.

\subsection{Personally Identifiable Information (PII)}

The concept of Personally Identifiable Information (PII) is central in privacy regulation. No PII in the collected data, then where is the harm in diclosing personal information. The scope of information requires privacy protection. Due to technological advances, it is somewhat easy to get some PII based on "public" information.

\section{Thursday, May 9th 2019}

\section{Week 3 - }

\section{Week 4 - }

\section{Week 5 - }

\section{Week 6 - }

\end{document}