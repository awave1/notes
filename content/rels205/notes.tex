\documentclass{article}
\usepackage[utf8]{inputenc}
\usepackage{hyperref}
\usepackage{graphicx}

\setlength{\parindent}{0em}
\setlength{\parskip}{1em}

\hypersetup{
    colorlinks,
    citecolor=black,
    filecolor=black,
    linkcolor=black,
    urlcolor=black
}

\title{RELS 205 - Religion and the Good Life}

\begin{document}

\maketitle
\tableofcontents

\section{Intro to Study of Religion}

\subsection{What is Religion?}

Religion is often thought to be associated with God. Religion is something communal, but it is privately held and important belief, it is both social and private. Religion has something to do with ethics.

There are varieties of religion phenomena, that can be seen in every religious tradition. There is social aspects to religion, such as communal worship. Personal aspects such as prayer, meditation, inward devotion, and religious experiences. Formal structures with buildings, institutions and clergy. Finally, there are volunturaly organizations with/without service focus. Religion can also be part of popular myth. For example, `the force' is a spiritual entity, similar to Brahman in Hinduism, Sophia in Judaism or the Logos in Christianity. Religion often has something to do with visual representation of the universe.

Objectivelly, religion is a social phenomenon, resulting from various cultural, historical, evolutionary and geographical accidents. It is universal type of social structure existing in particular circumstances. A particular stage in the evolution of human consciousness/societal development. The social \textbf{is the key} to understanding religion. The term itself \textbf{religion} comes from latin, \textit{religio}, an oath or bond; \textit{religare}, to bind together, a `religious' a member of monastic community. All of these have a social or public connatation.

Subjectively, religion is deeply help personal spirituality. The personal is the key to understanding religion, a set of personal beliefs and practices, what gives one's life meaning, one's emotional center.

\begin{center}
  \resizebox{\textwidth}{!}{
    \begin{tabular}{|c|c|}
      \hline
      \textbf{Negative} & \textbf{Positive} \\
      \hline
      \hline
      Set of falsities and crutches & one's necessary spirituality in life \\
      \hline
      A form of oppression & Reflective of true nature of universe \\
      \hline
      Stories developed over time for certain ends & Social structure that gives meaning to life \\
      \hline
    \end{tabular}
  }
\end{center}

Many theorists tried to define religion. Religion is a `floating signifier': a term whose meaning is different in each context in which it is find. In defining religion we have to reflect on our own expertise. \textbf{Religion is a way though which we organise the world}.

\textbf{Religion is a system of beliefs and practices} which are:

\begin{enumerate}
  \item Mythically illustrated (i.e. narrativized).
  \item Historically accreted. Religion is better seen as a `forest' rather than singular entity.
  \item Ritually performed.
  \item Communally negotiated, the community is the primary judges to what is religion or not.
  \item Personally appropriated. Religion will not continue if it wasn't personal.
  \item Institutionally articulated.
  \item Claiming license and authority from that about which it purports to speak.
  \item Aimed at achieving a vision of lasting human flourishing.
\end{enumerate}

We have now seen that religion is/has: subjective and objective elements, historical roots in communities and the oaths that bound one to them, constructed floating signifier. It is a categorization of the world and religion touches every aspect of human being.

\subsection{Concepts}

\subsubsection{God/Theos}

God is the concept that humans mediate or construct. The student of religion needs to be agnostic as to God's possible existence while taking the belief seriously. As a concept, we can understand how God works in a religion, but this is no different in essence from considering any other concept in this way.

The definition of religion doesn not only mark out a area of study, it also goes some way to defining the student. As religion is multifaceted, the student's approach must also be multifaceted. Religion therefore becomes a way of thinking about the universe through these different methodologies.

\subsubsection{Axiomatic analysis}

Axiom is a foundantional statement, true by definition or self-evidently. In religious studies axioms can be basic religious commitments (\textbf{ontologies}). Religious theorems are various beliefs and practices. Religious formulae are patterns of social life (morals, aesthetics). If patterns of social life are successful, then the axioms is positive is true at the beginning.

\begin{enumerate}
  \item phenomena - the world as it actually is
  \item model
  \item formulae
  \item theorem
  \item axiom
\end{enumerate}

\subsubsection{Theology}

Theology is defined as a grammar for religious language, belief and practice. Theology is as old as religion itself. Co-development of religious belief and practice with articulation of relevant and correct understanding of beliefs and practices.

\subsubsection{Philosophy}

Philosophy and theology were the same. Critical appraisal and desceription of essence and appearance of religious belief and practice. Eventually, philosophy becomes about the nature of the universe. Handmaiden to theology, eventually becomes independent. So-called neutral philosophy is still colored by religious categories and concepts.

\subsubsection{Anthropology}

Anthropology comes from undertanding ourselves in relation to the other. Study of human individuals and societies in their various physical, evolutionary and structural/relational forms.

\subsubsection{Sociology}

Sociology of religion considers religion as a function of or necessary factor for society.

\subsubsection{Psychology}

Psychology of religion tends to see religion as an outgrowth of psychic development and/or a factor in such developement.

\section{Religious language and experience}

\section{Faith and Reason}

\section{Suffering and Evil}

\section{Religion and Atheism}

\section{Religious diversity and pluralism}

\end{document}