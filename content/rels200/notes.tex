\documentclass{article}
\usepackage[utf8]{inputenc}
\usepackage{hyperref}
\usepackage[parfill]{parskip}

\hypersetup{
    colorlinks,
    citecolor=black,
    filecolor=black,
    linkcolor=black,
    urlcolor=black
}

\title{RELS 200 - Religious Myths and Worldviews}


\begin{document}

\maketitle
\tableofcontents
\newpage

\section{Module 1}

\subsection{What is Myth}

\subsubsection{What Mythology Is}

Mythology is literary and artistic works, that define achievments of nations and peoples. Mythology is often applied to some stories whose purpose is not limited to internatinment. Myths are narratives to explain and define people and they include society's values and perspective. Some stories are associated with a living religion \textbf{still being practiced at the time myth is told}. Myth is believed by many to be a `false story', which is related to popular use of myth to mean `false story'.

\subsubsection{What Mythology Is Not}

Mythological stories often contain elements that are not necessarily scientific, but the litiral falseness is important sometimes. Since myths are important to society, they are often handed down unchanged throughout many generations. Therefore, as a result, the scientific facts remain unchanged despite the time the myth has been told and people's understanding of the world.

Myths are \textbf{false} stories. However, they do contain literal falseness. Any false or outdated science that myths contain is not essential to the points myths are making about the nature of human beings and their role in the world. For example, stories about formation of cosmos can carry important truths to subsequent generations despite the fact that scientific views are outdated. Myths can tell about society's values and perspectives and can reveal true things about a culture.

\subsubsection{Alligators in the Sewers}

In modern society there exist so called \textbf{Urban legends}. These are not myths, but these are oral stories that still exist in our culture. No one knows why urban legends spread but they do have some elements of humour and surprise and they have a moral.

An urban legend about alligators in the sewers of New York originated in 1950s when it was popular to give out alligators as pets to kids. When alligarots grew big, they were simply flushed down the toilet and lived in the sewers. Some say that these alligarots are albino, due to the lack of sun and some also say they survive by eating rats. Some would argue that this story is a `myth' because it is false story that people would tell to explain a rumor. However the story is \textbf{just a legend} and has nothing to do with the actual myths. It shows however that people still pass oral stories through generations.

\subsubsection{Meaning of the Urban Legend}

The alligator story can be viewed more literally. In a city people are cutoff from nature. They do not walk on the ground, they walk on a layer that masks the nature thus separating the city life from the nature. The alligators represent fears of the unknown parts of the city, all the natural parts that we are not completely comfortable or familiar with.

\subsubsection{Myths and Legends as True Stories}

It is safe to say that story of alligarots if false but it does express some true concerns and conflicts of humans who live in cities. These concerns have been true of humans for many years, if the \textit{Epic of Gilgamesh} can be believed.

That does not make urban legen a myth. They do share some characteristics and they are related, but myths are about Gods and heroes, performing adventures. Urban Legends are oral stories that are reined through repetiotions. Mythology is made up of stories that are important to society and these stores become `fossilized'. Myths are not updated to include scientific discoveries.

\subsubsection{Functions and Insights}

Mythological stories represent truths about societies and cultures. Myths should not be considired as accurate representations of a society's values. There are two ways of looking at mythology. One approach is consider it from our point of view, thus seeing mythology \textbf{outside looking in}. Second way is to think about its role \textbf{inside a culture}. The stories told by societies have meaning and significance, called \textbf{functions of mythology}. Joseph Campbell identifies four main functions of mythology, \textbf{sociological}, \textbf{psychological}, \textbf{metaphysical}, and \textbf{cosmological}. Both functions and insights illumincate truth of mythological stories.

\subsubsection{Gods and Heroes}

Many understand myths as stories about gods and heroes. Such stories provide understanding of the nature of human life, for example Heracles. They bring metaphysical insights, what it means to be human and limitations of humans and psychological insights, struggles of individuals to become mature and useful members of society.

\subsubsection{Myth and Science}

Science and myths cover the same domain: the characterization of the natural world for the purpose of understanding and predicting the behavior of nature. Many think of myths as stories, stories to explain scientific factors. They also describe the origin of natural phenomena. This is called \textbf{aetiology} or \textbf{aetiological function}, from Greek word `cause'. Aetiology means to explain culture's understanding of the origin, cause of a custom or a fact of the physical universe, for example the raven is black because it flew through smoke according to Native American stories. It incorporates what was \textbf{the best scientific knowledge at the time}. Archaic societies used myths to explain the universe and why things were the way they were.

Cosmological insights captured by myths is how the universe was understood by the science available at the time.

\subsubsection{Trojan War}

\subsubsection{Insights Provided by Trojan War}

\subsection{Ways of understanding Myth}

\subsubsection{Who Studies Myth}

\subsubsection{Myths Have Many Versions}

\subsubsection{How to Read Myth}

\subsection{Folktale and Myth}

\subsection{Poetry and Myth}

\section{Module 2}
\section{Module 3}

\subsection{Destruction}

Myths of destruction represent a refinement on creation rather than simply the annihilation of a world or a mythological system. In the remaking of the world it is the characteristic and the essential that survives. The most predominant agent of destruction is \textbf{flood}. In \textit{Prose Edda}, the world is destroyed by \textbf{fire and warfare}.

\subsubsection{Destruction by Flood}

The story of the flood in the Western tradition originated in Mesopotamia, where flooding was common. The story was told to make different points. In Genesis, after the flood, God establishes covenant with people. In Ovid, the story tells that the new human beings are created from the rocks thrown by survivors of the flood.

\subsubsection{Destruction by Fire}

\textbf{Rangarok}, the end of the old world order, a total annihilation of all that existed before. At Rangarok, the earth will burn up, the gods and heroes will kill the monsters and they themselves will be killed in the battle.

\subsubsection{Rebirth: New Creation after the Destruction}

The destruction of the world is followed by a renewal of the earth and repopulation by righteous survivors or fresh new beginnings. In the Roman story, Pyrrha and Deucalion are spared because of their moral goodness, and from them, the new human race is to be born. In the \textit{Prose Edda}, new life emerges from the warm, moist earth. The heirs of Thor and Odin have survived. The sun has bourne a daughter. The personification of the sun allows the restoration of the world through her progeny and life proceeds. In Genesis and Metamorphoses, the destruction has taken place long time ago thus it serves us as a cautionary tale.

\section{Module 4}

\subsection{Heroes and Tricksters}

The hero is a figure whose accomplishments and adventures are recognized by a community as extraordinary. Heroes can serve humans as models, inspiration for success. The deeds of heroes vary from across different communities.

Sigmung Freud had made some influential discoveries about nature of hero stories. He was affected by the experiences and pathologies of his patients. For example, he discovered Oedipus complex, when a son strongly attracted to his mother is capable of directing substantial anger at his father for also wanting her attention.

Otto Rank was intrumental in developing the modern definition of the role of the hero in mythology. The myth of the hero is used to explain experiences that all humans undergo in their unconscious minds. According to his book \textit{The Myth of the Birth of the Hero}, number of stories from different cultures fit the pattern of the family romance. In the book, Rank uses the stories of these heroes to show that the pattern he identified was widespread enough to be considered universal. Since the heroes were not only individuals but also nation leaders, Rank's analysis not only consideres heroes as models of great deeds that might be emulated by the private citizens, but also their significance as embodying the values of a country or a religion.

According to Joseph Campbell's \textit{The Hero with a Thousand Faces}, heroes from every culture typically undertake a journey to a far-off land. In their travels they encounter villains, other heroes, and temptresses and often have the opportunity to bring back magical elixirs. Campbell also extends the concept of the hero's journey to the everyday life of the individual, explaining how the quest and its milestones mark our progress toward psychological wholeness.

\subsubsection{The Trickster: A Special Kind of Hero}

\section{Module 5}
\section{Module 6}

\end{document}
